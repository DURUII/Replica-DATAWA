\section{related work}
\label{VI}
Spatial Crowdsourcing (SC) is an innovative form of crowdsourcing that utilizes smart device carriers as workers who travel to specific locations to complete spatial tasks~\cite{zheng2022crowdsourced, 9474868, 9101624, 9101527, 8731525, zhao2023workerchurn, zhao2021fairness, zhao2019destination, zhao2020predictive, cheng2020real, deng2024task,yanwww2022,liu2022task,ChenZZ22,lai2022loyalty,zhao2023Coalition}.
SC can be categorized based on the task publication mode into Server Assigned Tasks (SAT) mode and Worker Selected Tasks (WST) mode.
In SAT mode, the server assigns tasks to nearby workers with the aim of optimizing system performance. This includes objectives such as maximizing the total number of tasks assigned~\cite{zhao2020predictive, zhao2019destination} or maximizing the overall payoff from these assignments~\cite{zhao2023workerchurn, zhao2021fairness}.
Research in this area has also explored factors like worker preference~\cite{zhao2023preference}, fairness~\cite{zhao2021fairness}, social networks~\cite{cheng2019event, lian2020geography, wang2019mcne}, and worker cooperation~\cite{li2023competition} in task assignment.
Ye et al.~\cite{ye2021task} propose a reinforcement learning based method to achieve the task allocation, and propose a graph neural network method with the attention mechanism to learn the embeddings of allocation centers, delivery points and workers.
Zhao et al.~\cite{zhao2023adataskrec} utilize graph convolutional network to model the geographical distance between out-of-town POIs, generating the out-of-town POI embeddings and to learn workers’ out-of-town preferences.
Wang et al.~\cite{Air-Ground} propose a novel communication-based multi-agent deep reinforcement learning method for data acquisition in urban environments.
Rao et al.~\cite{Can_You2024} propose an online framework by extending multi-agent reinforcement learning with careful augmentation to optimize the profit of order-serving and the data utility of crowdsensing.

However, most of the research conducted so far is based on the assumption of static offline scenarios, where the demand and supply between workers and tasks are known a prior. These studies focus on immediate task assignments, often struggling to adapt to the rapid and unpredictable changes in task demand and worker availability. This highlights the urgent need for further exploration and innovation in this field to ensure optimal task assignments over the long term.

However, SC is a real-time platform where workers and tasks occur dynamically. Recent studies consider task and worker prediction to solve online task assignment problems in SC. Zhai et al.~\cite{zhai2019seqst} propose a novel deep learning model to address the task prediction problem, which captures the temporal dependencies of historical task appearance in sequences at several time scales. Wei et al.~\cite{9816080} propose a location-and-preference joint prediction model to predict workers' locations and preference jointly at each timestamp. They also design a greedy multi-attribute joint task assignment algorithm to maximize the average number of completed tasks under constraints. Peng et al.~\cite{Peng2023} introduce a spatio-temporal prediction strategy that combines a gated recurrent unit and a variational autoencoder for crowdsourcing task prediction. However, the previous studies only concentrate on predicting task distribution, neglecting task demand dynamics, which is essential for accurately forecasting the spatio-temporal distribution of tasks.


% {\color{blue}and design a greedy multiattribute joint task assignment algorithm to maximize the average number of completed tasks under constraints.}

To solve the problem of task demand prediction, Yang et al.~\cite{yang2023batch} divide supply and demand into five degrees and use a Markov Predictor to predict the future degree of supply and demand. However, they do not consider the demand dependency in different regions, which is essential for predicting task demand. The closest related research to ours is~the work~\cite{zhao2020predictive}, which hybrids different learning models to predict the locations and routes of future workers and employs a graph embedding approach to estimate the distribution of future tasks but assigns each worker a fixed task sequence. However, in this work, we predict future task demands and update the assigned task sequence in real time for workers to ensure optimal task assignment over long time.

